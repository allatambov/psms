\documentclass[a4paper, 12pt]{article}
%%%%%%%%%%%%%%%%%%%%%%%%%%%%%%%%%%%%%%%%%%%%%%%%%%%%%%%%%%%%%%%%%%%%%%%%%%%%%%%%%%%%%%%%%%%%%%%%%%%%%%%%%%%%%%%%%%%%%%%%%%%%%%%%%%%%%%%%%%%%%%%%%%%%%%%%%%%%%%%%%%%%%%%%%%%%%%%%%%%%%%%%%%%%%%%%%%%%%%%%%%%%%%%%%%%%%%%%%%%%%%%%%%%%%%%%%%%%%%%%%%%%%%%%%%
% autocompile publish

\usepackage{math-hse}
\title{Дополнительные задачи}
\renewcommand{\thesubsection}{\arabic{subsection}}

\date{24.01.2020}
\begin{document}

\begin{problem}
Монету бросают $X$ раз -- до тех пор, пока хотя бы одна из её 
сторон, не обязательно подряд, не выпадет дважды. 
Составьте ряд распределения случайной величины $X$, 
вычислите $E(X)$.
\end{problem}

\begin{problem}
Пусть $X$ -- число решек, $Y$ -- число гербов, выпадающих при 
$n$ бросаниях монеты, $W = \max(X,Y)$. Вычислите $E(W)$ при $n=6$.
\end{problem}

\begin{problem}
Известно, что случайная величина $X$ принимает лишь натуральные значния, 
причём 

$$
P(X=n) = \frac{C}{n(n+1)}, n \in \mathbb{N}.
$$
Найдите: a) константу $C$; b) $P(X\leqslant10)$.
\end{problem}

\noindent\textit{Источник: Е.С.Кочетков, С.О.Смерчинская. 
Теория вероятностей в задачах и упражнениях. Москва. 2011.
}

\end{document}
