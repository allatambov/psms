\documentclass[a4paper, 12pt]{article}
%%%%%%%%%%%%%%%%%%%%%%%%%%%%%%%%%%%%%%%%%%%%%%%%%%%%%%%%%%%%%%%%%%%%%%%%%%%%%%%%%%%%%%%%%%%%%%%%%%%%%%%%%%%%%%%%%%%%%%%%%%%%%%%%%%%%%%%%%%%%%%%%%%%%%%%%%%%%%%%%%%%%%%%%%%%%%%%%%%%%%%%%%%%%%%%%%%%%%%%%%%%%%%%%%%%%%%%%%%%%%%%%%%%%%%%%%%%%%%%%%%%%%%%%%%
% autocompile publish

\usepackage{math-hse}
\title{Дискретные случайные величины: введение}
\renewcommand{\thesubsection}{\arabic{subsection}}

\date{24.01.2020}
\begin{document}

\begin{problem} 
Случайная величина $X$ принимает значение $(-1)$ в $30$\% случаев, 
$0$ -- в $25$\% случаев, $2$ -- в $15$\% случаев, $5$ -- в $12$\% случаев, 
$6$ -- в $18$\% случаев.  
\begin{itemize}
\item[a.] Постройте ряд распределения случайной величины $X$.
\item[b.] Найдите $P(X\leqslant0)$, $P(X\leqslant4.5)$, $P(X\leqslant6)$, $P(X<6)$, $P(X>7)$.
\item[c.] Найдите вероятность того, что $X$ принимает чётные значения.
\end{itemize}
\end{problem}

\begin{problem}
Дан ряд распределения случайной величины $X$:
\begin{center}
\begin{tabular}{|p{1cm}|p{1cm}|p{1cm}|p{1cm}|p{1cm}|p{1cm}|}
\hline
X & $-2$ & $-1$ & $0$ & $1$ & $2$\\
\hline
p & $0.3$ &$ $ & $0.2$  & $ $ & $0.1$\\
\hline
\end{tabular}
\end{center}
\begin{itemize}
\item[a.] Найдите пропущенные вероятности, если известно, 
что случайная величина $X$ принимает значения $-1$ и $1$ с равными вероятностями.
\item[b.] Запишите ряд распределения $X^2$. Запишите ряд распределения $X^3$.
\end{itemize}
\end{problem}

\begin{problem}
Дан ряд распределения случайной величины $Y$:
\begin{center}
\renewcommand*{\arraystretch}{1.3}
\begin{tabular}{|p{1cm}|p{1cm}|p{1cm}|p{1cm}|p{1cm}|p{1cm}|}
\hline
Y & 0 & 1 & 2 & 4\\
\hline
p & $1/2$ &$ $ & $1/6$  & $1/6$ \\
\hline
\end{tabular}
\end{center}
Найдите $F(-1)$, $F(1)$, $F(2.5)$, $F(4)$, $F(5.5)$, где $F$ -- 
функция распределения случайной величины $Y$. 
\end{problem}

\begin{problem}
Дан ряд распределения случайной величины $X$:
\begin{center}
\renewcommand*{\arraystretch}{1.3}
\begin{tabular}{|p{1cm}|p{1cm}|p{1cm}|p{1cm}|p{1cm}|}
\hline
X & $-1$ & $0$ & $1$ & $3$\\
\hline
p & $1/5$ & $2/5$  & $1/5$  & $1/5$ \\
\hline
\end{tabular}
\end{center}
Найдите математическое ожидание случайной величины $X$. 
Найдите математическое ожидание случайной величины $X^2$.
\end{problem}

\begin{problem}
Гарри сидит за столом в Большом Зале, завтракает и ждет почту. 
С вероятностью $0.2$ ему может прийти письмо от профессора МакГонагалл, 
с вероятностью $0.7$ -- от Хагрида. Известно, что МакГонагалл и Хагрид 
действуют независимо. Постройте ряд распределения числа полученных Гарри 
писем и найдите его  математическое ожидание.
\end{problem}

\begin{problem}
На избирательном участке зарегистрировано три избирателя. Вероятность того, что первый
из них пойдёт на выборы, равна $0.6$, у второго эта вероятность -- $0.5$, 
а у третьего -- $0.2$. Избиратели принимают решение об участии в выборах независимо.
Постройте ряд распределения явки на этом участке. Найдите математическое ожидание явки.\footnote{
А.А.Макаров, А.В.Пашкевич, А.А.Тамбовцева. Задачник по математической статистике для студентов социально-гуманитарных и управленческих специальностей. 2018.}
\end{problem}
\end{document}
